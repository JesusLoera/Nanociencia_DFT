% Elaborado por Jesús Loera

\documentclass[12pt]{article}
\usepackage{graphicx}
\usepackage{scrextend}
\usepackage{vmargin}
\usepackage[utf8]{inputenc}
\usepackage[spanish,es-tabla]{babel}
\usepackage{subcaption}
\usepackage{caption}
\usepackage{multicol}
\usepackage{multirow}
\usepackage{breakcites}
\usepackage[breaklinks=true]{hyperref}
\usepackage{amsmath, amsthm, amssymb, amsfonts}
\usepackage[usenames]{color}
\usepackage[table,xcdraw]{xcolor}
\usepackage{float}

\usepackage[nottoc]{tocbibind} %Adds "References" to the table of contents

\usepackage{mathrsfs,scalerel}  % Laplace


\newsavebox\foobox
\newlength{\foodim}
\newcommand{\slantbox}[2][0]{\mbox{%
        \sbox{\foobox}{#2}%
        \foodim=#1\wd\foobox
        \hskip \wd\foobox
        \hskip -0.5\foodim
        \pdfsave
        \pdfsetmatrix{1 0 #1 1}%
        \llap{\usebox{\foobox}}%
        \pdfrestore
        \hskip 0.5\foodim
}}
\def\Laplace{\ThisStyle{\slantbox[-.45]{$\SavedStyle\mathscr{L}$}}}


\renewcommand{\abstractname}{Descripción}   % Cambiamos el nombre del abstract
\renewcommand{\contentsname}{Contenidos}   % Cambiamos el nombre de contents

\parindent=0mm
\pagestyle{empty}
\definecolor{citecolor}{rgb}{.12,.54,.11}
\definecolor{urlcolor}{RGB}{64, 145, 108}
\definecolor{1ee592}{RGB}{30,229,146}
\definecolor{1bac70}{RGB}{27,172,112}
\definecolor{f38638}{RGB}{243,134,56}
\definecolor{black}{RGB}{0,0,0}
\definecolor{gray}{RGB}{156,156,156}
\hypersetup{
    colorlinks=true,
    linkcolor=blue,
    filecolor=magenta,
    urlcolor=urlcolor,
    citecolor=citecolor,
}

\author{Jesus Eduardo Loera Casas}
\title{Clase} 
\subtitle{Resolviendo el oscilador forzado amortiguado}
\institute{
    Universidad Autónoma de Nuevo León\\
    Facultad de Ciencias Físico Matemáticas
} 
\date{01 de Diciembre del 2020} 
\useoutertheme{infolines}
%\setbeamercovered{transparent} 
%\setbeamertemplate{navigation symbols}{} 
%\logo{} 
%\subject{} 

\AtBeginSection{

    \begin{frame}<beamer>{Contenido}
        \tableofcontents[currentsection]
    \end{frame}
}

\begin{document}

\begin{titlepage}
    \centering
    \vspace{1cm}
    {\bfseries\LARGE Universidad Autonoma de Nuevo León \par}
    \vspace{1cm}
    {\scshape\Large Facultad de Ciencias Físico Matemáticas \par}
    \vspace{3cm}
    {\scshape\Huge Reporte 1  \par}
    \vspace{3cm}
    {\scshape\Huge Silicio, Siliceno y Silicano  \par}
    \vspace{3cm}
    {\Large Autor: \par}
    \vfill
    {\Large Jesús Eduardo Loera Casas 1898887 \par}
    {\Large \today \par}
\end{titlepage}

\tableofcontents			% Índice
\newpage

\begin{center}
	\rule[0mm]{150mm}{0.1mm}		% Para dibujar una linea horizontal de
									% [elevación]{longitud}{grosor}
	\end{center}
	
	
\begin{abstract}		% ABSTRACT
  
\noindent 				% Anula la sangria de este parrafo

En este documento se presenta la actividad 1 del proyecto de investigación en Nanociencia,
usando la Teoría Funcional de la Densidad (DFT). 
En este mismo se reportan los resultados de las simulaciones de una serie de cálculos 
realizados con el silicio y el siliceno con el fin de describir las características electrónicas de sus
celdas unitarias; ademas se reportan datos recabados en la literatura de dichos 
materiales además del silicano.


Se econtraron los parametros de red del silicio, siliceno y silicano, los cálculos mostraron
que se encontraban cercanos a 5.4, 3.8 y 3.9 amstrongs, respectivamente. 
Ádemas se hicieron cálculos de bandas y densidad de estados para posteriormente 
gráficarlos.
\end{abstract}
	
\begin{center}
	\rule[0mm]{150mm}{0.1mm}
	\end{center}

\section{Actividad}	
    \textbf{Simulación estructural y electrónica del silicio.}

\vspace{0.4cm}

Objetivos:

\begin{itemize}
    \small
    \item Optimización de ecutwfc
    \item Optimización de ecutrho
    \item Optimización de puntos k
    \item Optimización de parámetro de red
\end{itemize}

\vspace{0.4cm}

\textbf{Simulación estructural y electrónica del Siliceno.}

\vspace{0.4cm}

Objetivos:

\begin{itemize}
    \small
    \item Optimización de puntos k
    \item Optimización de parámetro de red
    \item Cálculo de bandas
    \item Densidad de estados
\end{itemize}

\textbf{Simulación estructural y electrónica del Silicano.}

\vspace{0.4cm}

Objetivos:

\begin{itemize}
    \small
    \item Optimización de ecutwfc
    \item Optimización de ecutrho
    \item Optimización de puntos k
    \item Optimización de parámetro de red
    \item Cálculo de bandas
    \item Densidad de estados
\end{itemize}

\vspace{0.4cm}

\textbf{ Silicio, siliceno y silicano.}

\begin{itemize}
    \item Hacer una búsqueda en la literatura (artículos cientificos) de los parámetros 
          estructurales (parámetro de red, longitud de enlace, ángulos) y propiedades
          electrónicas (estructura de bandas electrónicas, densidad de estados electrónicos)
          del silicio, siliceno y silicano.
\end{itemize}  
    
\section{Métodos computacionales}
    
Para el desarrollo de esta actividad, usamos la Teoría Funcional de la Densidad (DFT)
y el software Quantum Espresso, los cuales nos sirvieron para estudiar los materiales
previamente mencionados.

Quantum Espresso es un software programado en Fortran y C, cuyo código es gratis y distribuido por la Licencia
Pública General GNU, usa ondas planas como funciones base y pseudopotenciales para describir las 
interacciones entre iones y electrones. \cite{WinNT}

\vspace{0.5cm}

El cuál nos permite, entre otras cosas, realizar los siguientes cálculos: \cite{WinNT}

\begin{itemize}
    \item Cálculos para el estado base
    \item Optimización estructural 
    \item Estados de transición de caminos de mínima energía
    \item Dinámica molecular ab initio
    \item Transporte cuántico
    \item Propiedades espectroscópicas
\end{itemize}

\vspace{1cm}

\begin{figure}[H]
    \centering
    \includegraphics[scale=0.5]{logo_header.jpg}
    \caption{Quantum Espresso, software utilizado para realizar el presente trabajo. Hoy en día 
             es una herramiento muy vérsatil para investigadores que se adentran en la ciencia 
             de materiales.}
\end{figure}

\newpage


\section{Pseudopotenciales}
    
[PSEUDO] Se utilizaron los pseudopotenciales $H.pbe-rrkjus_psl.1.0.0.UPF$ y 
$Si.pbe-n-rrkjus_psl.1.0.0.UPF$ de $http://www.quantum-espresso.org.$

\vspace{0.5cm}

\textbf{Hidrógeno}

\begin{itemize}
    \item USPP
    \item PBE 
    \item Scalar relativistic
\end{itemize}

\vspace{0.5cm}

\textbf{Silicio}

\begin{itemize}
    \item USPP
    \item PBE 
    \item Non linear Core Correction
    \item Scalar relativistic
\end{itemize}

\newpage
                            
\section{Simulación estructural y electrónica del silicio.}	
    \section{Silicio}

\begin{frame}{Silicio}
    \begin{figure}[H]
        \centering
        \includegraphics[scale=0.2]{images/longitud_enlace_5_4700_amstrongs.png}
        \caption{El parámetro de red es de 5.4700 amstrongs.}
    \end{figure}
\end{frame}  
    
\section{Simulación estructural y electrónica del siliceno.}	
    \section{Siliceno}


\begin{frame}{Siliceno}

    \textbf{Párametro de red}

    \begin{figure}[H]
        \centering
        \includegraphics[scale=0.2]{images_siliceno/longitud_enlace_3_8600_amstrongs.png}
        \caption{La longitud del enlace mostrado es de 3.8600 amstrongs.}
    \end{figure}
\end{frame}


\begin{frame}

    \textbf{Estructura electrónica de bandas (sin considerar el spin)}

    \begin{figure}[H]
        \centering
        \includegraphics[scale=0.34]{images_siliceno/bands_structure.png}
        \caption{Estructura de bandas del Siliceno.}
    \end{figure}
\end{frame}

\begin{frame}
    \begin{figure}[H]
        \centering
        \includegraphics[scale=0.25]{images_siliceno/plot_sin_spin.png}
        \caption{Gráfica que nos muestra la densidad de estados del siliceno, sin considerar el spin.}
    \end{figure}
\end{frame}

\begin{frame}
    \begin{figure}[H]
        \centering
        \includegraphics[scale=0.3]{images_siliceno/siliceno_energía_orbitales_up_sin_spin.png}
        \caption{Gráfica que nos muestra la densidad de estados en los orbitales del siliceno, sin considerar el spin [UP].}
    \end{figure}
\end{frame}

\begin{frame}
    \begin{figure}[H]
        \centering
        \includegraphics[scale=0.3]{images_siliceno/siliceno_energía_orbitales_down_sin_spin.png}
        \caption{Gráfica que nos muestra la densidad de estados en los orbitales del siliceno, sin considerar el spin [DOWN].}
    \end{figure}
\end{frame}


\begin{frame}

    \textbf{Estructura electrónica de bandas (considerando el spin)}

    \begin{figure}[H]
        \centering
        \includegraphics[scale=0.25]{images_siliceno/densidad_estados_con_spin.png}
        \caption{Gráfica que nos muestra la densidad de estados del siliceno, considerando el spin.}
    \end{figure}
    
\end{frame}

\begin{frame}
    \begin{figure}[H]
        \centering
        \includegraphics[scale=0.3]{images_siliceno/silicio_energía_orbitales_up.png}
        \caption{Gráfica que nos muestra la densidad de estados en los orbitales del siliceno, considerando el spin [UP].}
    \end{figure}
\end{frame}

\begin{frame}
    \begin{figure}[H]
        \centering
        \includegraphics[scale=0.3]{images_siliceno/siliceno_energía_orbitales_down.png}
        \caption{Gráfica que nos muestra la densidad de estados en los orbitales del siliceno, considerando el spin [DOWN].}
    \end{figure}
\end{frame}     
    
\section{Datos estructurales del silicio, siliceno y silicano.}	
    
\vspace{0.5cm}

El siliceno hidrogenado es conocido como el silicio, tiene una estructura similar al siliceno pero 
con átomos de hidrógeno alternando en el plano formando una hibridación $sp^{3}$.

\vspace{0.5cm}

\begin{figure}[H]
    \centering
    \includegraphics[scale=0.3]{images_silicano/silicano_structure.png}
    \caption{Estructura cristalina del Siliceno obtenida del archivo input con Xcrysden [Celda primitiva]}
\end{figure}

\vspace{0.5cm}

\begin{figure}[H]
    \centering
    \includegraphics[scale=0.3]{images_silicano/silicano_structure_2.png}
    \caption{Estructura cristalina del Siliceno (otra perspectiva) [Celda primitiva]}
\end{figure}

\begin{table}[H]
    \begin{center}
        \begin{tabular}{| c | c |}
            \hline
            \multicolumn{2}{ |c| }{\textbf{Archivo inicial}} \\ \hline
            ibrav & 0 \\ \hline
            nat & 4 \\ \hline
            ntype & 2 \\ \hline
            occupations & smearing \\ \hline
            degauss & 0.01 \\ \hline
            smearing & 'm-p' \\ \hline
            Cell parameters  {alat}  & 0.500 0.866 0.000  \\
                                     & -0.500 0.866 0.000 \\
                                     & 0.000 0.000 5.144  \\ \hline
            Atomic Positions Crystal & H  0.000 0.000 0.000  \\
                                     & Si 0.000 0.000 0.075  \\
                                     & Si 0.333 0.333 0.111  \\
                                     & H  0.333 0.333 0.186 \\  \hline
        \end{tabular}
        \caption{Principales paramétros del silicano}
        \label{tab: Parametros del Silicano}
    \end{center}
\end{table}

A continuación, realizaremos los siguientes cálculos y optimizaciones:

\begin{itemize}
    \item Optimización Ecutwfc [Silicano]
    \item Optimización Ecutrho [Silicano]
    \item Optimización K-points [Silicano]
    \item Optimización Lattice parameter (parámetro de red) [Silicano]
    \item Cálculo de bandas [Silicano]
    \item Densidad de estados sin considerar el spin [Silicano]
    \item Densidad de estados considerando el spin [Silicano]
\end{itemize}


% Optimización Ecutwfc [Silicano]

\newpage

\subsection{Optimización Ecutwfc [Silicano]}

\vspace{0.5cm}

Buscamos obtener un valor mínimo para el parámetro Ecutwfc donde podamos observar 
que converge la energía de la estructura cristalina.

\vspace{0.5cm}

Vamos a variar el valor del paramétro Ecutwfc de 20 a 50 en pasos de 5 en 5, así realizaremos un cálculo de
SCF y un ploteo de Ecutwfc contra la energía total de la estructura.

\vspace{0.5cm}

\begin{figure}[H]
    \centering
    \includegraphics[scale=0.32]{images_silicano/Ecutwfc_vs_Energy.png}
    \caption{Gráfica que nos muestra la energía total del sistema contra la variación del parámetro ecutwfc.}
\end{figure}


Observación: el valor de la variable dependiente (la energía total) empieza su convergencia cuando 
la variable independiente (ecutwfc) empieza a tomar valores a partir de 35.


% Optimización Ecutrho [Silicano]

\newpage

\subsection{Optimización Ecutrho [Silicano]}

\vspace{0.5cm}

Ahora vamos a buscar optimizar el valor del paramétro Ecutrho, donde para fines de nuestra actividad 
supondremos que es un múltiplo de el valor que toma Ecutwfc, donde depende de la siguiente forma:

\begin{equation*}
    Ecutrho = k*Ecutwfc
\end{equation*}

Donde $ k \in \mathbb{N}  $

\vspace{0.5cm}

Buscamos obtener un valor mínimo para k de tal manera que podamos observar que la energía de la
estructura cristalina converge.

\vspace{0.5cm}

Vamos a variar el valor de k de 1 a 12 en pasos de 1 en 1, así realizaremos un cálculo de
SCF y un ploteo del valor de k contra la energía total de la estructura.

\begin{figure}[H]
    \centering
    \includegraphics[scale=0.33]{images_silicano/K_in_ecutrho_vs_Energy.png}
    \caption{Gráfica que nos muestra la energía total del sistema contra la variación del parámetro k, nos centramos en el intervalo [4,12]}
\end{figure}

Observación: el valor de la variable dependiente (la energía total) empieza su convergencia cuando 
la variable independiente (ecutwfc) empieza a tomar valores a partir de 6.


% Optimización K-points [Silicano]

\newpage

\subsection{Optimización K-points [Silicano]}

\vspace{0.5cm}

A continuación realizaremos una optimización para los puntos K del silicano. En nuestro archivo input estos
tienen la forma de la siguiente expresión: 

\begin{equation*}
    i \; i \; 1 \; 0 \; 0 \; 0
\end{equation*}

Donde $ i \in \mathbb{N}  $ .

\vspace{0.5cm}

Buscamos obtener un valor mínimo para i de tal manera que podamos observar una convergencia en la 
energía del sistema.

\begin{figure}[H]
    \centering
    \includegraphics[scale=0.33]{images_silicano/K_points_vs_Energy.png}
    \caption{Gráfica que nos muestra la energía total del sistema contra la variación del parámetro k, nos centramos en el intervalo [4,12]}
\end{figure}

\vspace{0.5cm}

Observación: el valor de la variable dependiente (la energía total) empieza su convergencia cuando 
la variable independiente (i) empieza a tomar valores alrededor de 8.



% Optimización Lattice parameter (parámetro de red) [Silicano]

\newpage

\subsection{Optimización Lattice parameter (parámetro de red) [Silicano]}

\vspace{0.5cm}

Ahora vamos a buscar optimizar el valor del paramétro de red, haremos una inspección probando valores
en el siguiente intervalo [3.0, 5.0].
 
\vspace{0.5cm}

Buscamos obtener un valor mínimo para el paramétro de red de tal manera que podamos observar que la energía
de la estructura cristalina converge.

\vspace{0.5cm}

Empezaremos desde el 3.0 y para cada uno haremos un calculo del tipo "relax", repetiremos el proceso
en saltos de 0.1 en 0.1 hasta alcanzar el valor de 5.0.

\begin{figure}[H]
    \centering
    \includegraphics[scale=0.33]{images_silicano/Lattice_parameter_vs_Energy.png}
    \caption{Gráfica que nos muestra la energía total del sistema contra la variación del parámetro k, nos centramos en el intervalo [3.0, 5.0]}
\end{figure}

\vspace{0.5cm}

Observación:,La energía se minimiza cuando el parámetro de red empieza a tomar valores entre 3.85 y 3.95,
aproximadamente.

\vspace{0.5cm}

Vamos a realizar una segunda inspección, empezaremos en el 3.85 y terminaremos en 3.95, dando saltos
de 0.01 en 0.01. Esto nos permitirá determinar otra cifra de precisión.

\begin{figure}[H]
    \centering
    \includegraphics[scale=0.42]{images_silicano/Lattice_parameter_vs_Energy_second.png}
    \caption{Gráfica que nos muestra la energía total del sistema contra la variación del parámetro red, nos centramos en el intervalo [3.85, 3.95]}
\end{figure}

\noindent
Observación: La energía se minimiza cuando el parámetro de red toma el valor de 3.9.

\vspace{0.5cm}

Realizaremos una tercera inspección, empezaremos en el 3.885 y terminaremos en 3.895, dando saltos
de 0.001 en 0.001. Esto nos permitirá determinar otra cifra de precisión.

\begin{figure}[H]
    \centering
    \includegraphics[scale=0.42]{images_silicano/Lattice_parameter_vs_Energy_third.png}
    \caption{Gráfica que nos muestra la energía total del sistema contra la variación del parámetro red, nos centramos en el intervalo [3.885, 3.895]}
\end{figure}

\noindent
Observación: La energía se minimiza cuando el parámetro de red toma el valor de 3.92.

\vspace{0.5cm}

\noindent
Ahora vamos a mostrar la estructura cristalina ya relajada.

\vspace{0.5cm}

\begin{figure}[H]
    \centering
    \includegraphics[scale=0.30]{images_silicano/Longitud_enlace_H_Si_1_5015_amstrongs.png}
    \caption{La longitud del enlace mostrado es de 1.5015 amstrongs.}
\end{figure}

\vspace{0.5cm}

\begin{figure}[H]
    \centering
    \includegraphics[scale=0.34]{images_silicano/Longitud_enlace_Si_Si_2_3576_amstrongs.png}
    \caption{La longitud del enlace mostrado es de 2.3576 amstrongs.}
\end{figure}

\vspace{0.5cm}

\begin{figure}[H]
    \centering
    \includegraphics[scale=0.34]{images_silicano/Longitud_enlace_Si_Si_4_5504_amstrongs.png}
    \caption{La longitud del enlace mostrado es de 4.5504 amstrongs.}
\end{figure}


% Cálculo de bandas [Silicano]

\newpage

\subsection{Cálculo de bandas [Silicano]}

En el siguiente apartado damos una gráfica con la estructura electrónica de bandas del silicano. 
Para obtener esta estructura es necesario realizar la siguiente serie de cálculos.
Primero realizamos un cálculo con el módulo pw.x, después otro cálculo 
del tipo scf y por último uno del tipo bands. Se llegó al siguiente gráfico:

\begin{figure}[H]
    \centering
    \includegraphics[scale=0.45]{images_silicano/bands_structure_silicane_10_bands_relax.png}
    \caption{Gráfica que nos muestra la estructura de bandas del siliceno, cuyo cálculo fue elaborado por cuenta propia}
\end{figure}

En la literatura podemos encontrar estructuras similares, como por ejemplo:

\begin{figure}[H]
    \centering
    \includegraphics[scale=0.5]{images_silicano/Bandstructure-of-a-silicane-and-b-germanane.png}
    \caption{Gráfica que nos muestra la estructura de bandas del silicano [a)] \cite{trivedi2014silicene} }
\end{figure}

\vspace{0.2cm}

Observación: Ambas son semejantes. Se observan discrepencias,
sin embargo, cabe la posibilidad que sea debido al poder de computo del que se dispone.


% Densidad de estados sin considerar el spin [Silicano]

\newpage

\subsection{Densidad de estados sin considerar spin [Silicano]}

En el siguiente apartado damos una gráfica con la densidad de estados del siliceno. Para 
ello se hizo el cálculo dentro del mismo Quantum Espresso con el módulo pw.x, primero se hizo uno cálculo 
del tipo nscf y después otro del tipo DOS. También para este momento se trabajó con la estructura
ya relajada.

\vspace{0.5cm}

\textbf{Importante: En el siguiente cálculo no se consideró el spin.}

\vspace{0.5cm}

Se llegó al siguiente gráfico:

\begin{figure}[H]
    \centering
    \includegraphics[scale=0.35]{images_silicano/Densidades_estado_sin_spin.png}
    \caption{Gráfica que nos muestra la densidad de estados del siliceno, sin considerar el spin.}
\end{figure}

A continuación proporcionaremos una gráfica que nos muestra la 
densidad de estados por nivel orbital.

\begin{figure}[H]
    \centering
    \includegraphics[scale=0.28]{images_silicano/Densidad_estados_sin_spin_up.png}
    \caption{Gráfica que nos muestra la densidad de estados en los orbitales del silicano, sin considerar el spin. [UP]}
\end{figure}

\begin{figure}[H]
    \centering
    \includegraphics[scale=0.38]{images_silicano/Densidad_estados_sin_spin_down.png}
    \caption{Gráfica que nos muestra la densidad de estados en los orbitales del silicano, sin considerar el spin. [DOWN]}
\end{figure}

\vspace{0.5cm}

A continuación proporcionaremos una gráfica que nos muestra la densidad de 
estados por elemento.

\begin{figure}[H]
    \centering
    \includegraphics[scale=0.38]{images_silicano/Densidad_estados_sin_spin_up_elementos.png}
    \caption{Gráfica que nos muestra la densidad de estados por elemento, sin considerar el spin. [UP]}
\end{figure}

\begin{figure}[H]
    \centering
    \includegraphics[scale=0.38]{images_silicano/Densidad_estados_sin_spin_down_elementos.png}
    \caption{Gráfica que nos muestra la densidad de estados por elemento, sin considerar el spin. [DOWN]}
\end{figure}

% Densidad de estados considerando el spin [Silicano]

\newpage

\subsection{Densidad de estados considerando el spin [Silicano]}

En el siguiente apartado damos una gráfica con la densidad de estados del siliceno. Para 
ello se hizo el cálculo dentro del mismo Quantum Espresso con el módulo pw.x, primero se hizo uno cálculo 
del tipo nscf y después otro del tipo DOS. También para este momento se trabajó con la estructura
ya relajada.

\vspace{0.5cm}

\textbf{Importante: En el siguiente cálculo se consideró el spin.}

\vspace{0.5cm}

Se llegó al siguiente gráfico:

\begin{figure}[H]
    \centering
    \includegraphics[scale=0.33]{images_siliceno/densidad_estados_con_spin.png}
    \caption{Gráfica que nos muestra la densidad de estados del siliceno, considerando el spin.}
\end{figure}

A continuación proporcionaremos una gráfica que nos muestra la energía por nivel orbital.

\begin{figure}[H]
    \centering
    \includegraphics[scale=0.3]{images_silicano/Densidad_estados_con_spin_up_1s.png}
    \caption{Gráfica que nos muestra la densidad de estados en los orbitales del silicano, considerando el spin. [UP]}
\end{figure}

\begin{figure}[H]
    \centering
    \includegraphics[scale=0.38]{images_silicano/Densidad_estados_con_spin_down.png}
    \caption{Gráfica que nos muestra la densidad de estados en los orbitales del silicano, considerando el spin. [DOWN]}
\end{figure}

\vspace{0.5cm}

A continuación proporcionaremos una gráfica que nos muestra la energía por elemento.

\begin{figure}[H]
    \centering
    \includegraphics[scale=0.38]{images_silicano/Densidad_estados_con_spin_up_elementos.png}
    \caption{Gráfica que nos muestra la densidad de estados por elemento, considerando el spin. [UP]}
\end{figure}

\begin{figure}[H]
    \centering
    \includegraphics[scale=0.38]{images_silicano/Densidad_estados_con_spin_down_elementos.png}
    \caption{Gráfica que nos muestra la densidad de estados por elemento, considerando el spin. [DOWN]}
\end{figure}

\newpage      

\newpage    
\bibliographystyle{unsrt}
\nocite{*}
\bibliography{Referencias}

\end{document}