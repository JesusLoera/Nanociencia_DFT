
Para el desarrollo de esta actividad, usamos la Teoría Funcional de la Densidad (DFT)
y el software Quantum Espresso, los cuales nos sirvieron para estudiar los materiales
previamente mencionados.

Quantum Espresso es un software programado en Fortran y C, cuyo código es gratis y distribuido por la Licencia
Pública General GNU, usa ondas planas como funciones base y pseudopotenciales para describir las 
interacciones entre iones y electrones. \cite{WinNT}

\vspace{0.5cm}

El cuál nos permite, entre otras cosas, realizar los siguientes cálculos: \cite{WinNT}

\begin{itemize}
    \item Cálculos para el estado base
    \item Optimización estructural 
    \item Estados de transición de caminos de mínima energía
    \item Dinámica molecular ab initio
    \item Transporte cuántico
    \item Propiedades espectroscópicas
\end{itemize}

\vspace{1cm}

\begin{figure}[H]
    \centering
    \includegraphics[scale=0.5]{logo_header.jpg}
    \caption{Quantum Espresso, software utilizado para realizar el presente trabajo. Hoy en día 
             es una herramiento muy vérsatil para investigadores que se adentran en la ciencia 
             de materiales.}
\end{figure}

\newpage
