
\section{Resumen}

\begin{frame}{Resumen}
    En esta presentación se reportan los resultados la actividad 1 del proyecto de investigación en 
    Nanociencia, que tiene como soporte la Teoría Funcional de la Densidad (DFT) y se usó Quantum
    Espresso para su desarrollo.

    \vspace{0.5cm}

    \begin{figure}[H]
        \centering
        \includegraphics[scale=0.45]{logo_header.jpg}
        \caption{Quantum Espresso, software utilizado para realizar el presente trabajo. Hoy en día 
                 es una herramienta muy vérsatil para investigadores que se adentran en la ciencia 
                 de materiales.}
    \end{figure}
\end{frame}

\begin{frame}
    \begin{block}{Resultados}
        Se encontraron los parametros de red del silicio, siliceno y silicano, los cálculos mostraron
        que se encontraban cercanos a 5.47, 3.892 y 3.92 amstrongs, respectivamente. 
        Ádemas se hicieron cálculos de bandas y densidad de estados para el siliceno y el silicano.
    \end{block}
\end{frame}



